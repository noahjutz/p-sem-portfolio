\documentclass[../../main.tex]{subfiles}

In Vorbereitung auf mein Bewerbungsschreiben zähle ich im folgenem Text meine Stärken und Schwächen auf.
Zu meinen Stärken gehören sprachliche, technische und körperliche Fähigkeiten.
Hinsichtlich der sprachlichen Fähigkeiten weise ich sowohl rudimentäre Spanischkenntnisse, als auch flüssiges Englisch auf.
Zum körperlichen lässt sich anmerken, dass ich seit einem Jahr konsequent Muskelaufbautraining betreibe. Dazu kommt, dass ich mittlerweile dehnbar genug bin, um einen Spagat zu machen. Beides erfordert viel Disziplin.
Blicken wir zuletzt auf meinen Fähigkeitsschwerpunkt: Meine weitgehende technische Kentnisse. Zu diesen gehört u. A. der Umgang mit Hardware: so habe ich meinen eigenen PC gebaut, und im Rahmen meiner W-Seminararbeit einen Arduino verwendet. Noch viel besser kenne ich mich aber mit dem Softwaretechnischem aus; als Systemadministrator eines Linux-Servers war ich u. A. Host meines eigenem Apache-Servers. Was tatsächliches Programmieren angeht, habe ich grundlegende Kenntnisse einiger Bereiche - ob Web-Design mit HTML, CSS und EMCAScript, Skripte mit Python und Bash, oder auch die Entwicklung von Android-Apps mit Kotlin.
Was Social-Skills angeht, bin ich kein extrovertierter, gesprächiger Mensch, aber ein guter Zuhörer. Ich halte es für schwierig, Gespräche mit fremden Menschen zu halten. Schulisch sind meine Leistungen mittelmäßig; daraus lässt sich folgern, dass ich unzureichende Disziplin in Bereichen habe, welche mir keinen unmittelbaren Vorteil bringen und keinen Spaß machen. Ich bin auch kein flexibler Mensch, weil ich keine Kompromisse im Training eingehe, und mir ein regelmäßiger Tagesablauf wichtig ist.