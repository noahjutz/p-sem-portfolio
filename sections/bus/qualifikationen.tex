\documentclass[../../main.tex]{subfiles}

Nachdem ich meine Stärken und Schwächen in SELBSTEINSCHÄTZUNG analysiert habe, konzentrieren wir uns nun auf die Stärken - genauer: meine konkreten Qualifikationen.
Neben meinem muttersprachlichem Deutsch spreche ich auch fließend Englisch und habe grundlegende Spanischkenntnisse.
Ich erhalte im Laufe des Jahres mein Abitur.
Schwerpunkt meiner Qualifikationen sind jedoch meine technischen Kenntnisse, spezifisch im Bereich Programmieren. Meine Fertigkeiten hinsichtlich logischem Denkens, Programmiersprachlich-syntaktischem Wissens und Umgang mit sämtlichen APIs wiederspiegeln sich in meinen Projekten. So schrieb ich meine W-Seminararbeit in TeX [1], und erstellte ein Python-Skript, um den Fortschritt zu visualisieren [2]. Rudimentäre Web-Kenntnisse zeigen sich darin, dass ich eine eigene Website erstellt habe [3]. Aber im Mittelpunkt meiner Qualifikationen stehen meine Android-Kenntnisse. "GymRoutines" [4] ist eine Android-App, geschrieben in Kotlin, welche einige meiner zentralen Stärken unter Beweist stellt: SQLite-Datenbank-management mit Jetpack Room, Benutzeroberflächendesign mit Jetpack Compose, Test-driven-development, Continuous Integration, Dependency Injection, MVVM-Architektur, und mehr.
Schlussendlich lässt sich erkennen, dass mein Kompetenzenschwerpunkt bei Android-App-Entwicklung liegt. Meine Qualifikationen umfassen sowohl sprachliche und technische Kenntnisse, als auch schulische Leistungen.

[1] github.com/noahjutz/w-seminararbeit
[2] github.com/noahjutz/texlapse
[3] noahjutz.neocities.org
[4] github.com/noahjutz/GymRoutines