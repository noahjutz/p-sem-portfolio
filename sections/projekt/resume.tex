\documentclass[../../main.tex]{subfiles}

Die Teilhabe an der Entwicklung der Videospielgestaltung in der neuen Hedwigsklinink war für mich persönlich eine lehrreiche Erfahrung. Motivation für die Wahl dieses P-Seminares war die Möglichkeit, meine Digital Design Skills anzuwenden. Vor Allem das 3D-Design mit Blender war für mich Interessant. Darüber hinaus ist das Gestalten einer Klinik als Ziel für mich sehr wichtig.

Die Umfrage, welche ich unternommen habe, war für mich die wichtigste Informationsquelle und hatte einen bedeutenden Einfluss auf meine Entscheidungen. Ich habe zum ersten mal die Erfahrung gemacht, eine Umfrage zu erstellen und auszuwerten. Meine Konzepte, darunter das Konsolendesign, das Kabelmanagement, und die Benutzer-Schaltoberfläche, habe ich mit Blender visualisiert. Für die Visualisierung meiner Konzepte habe ich gelernt, wie man diese 3D-Software verwendet. Im Abschnitt der Berufs- und Studienorientierung konnte ich meine Zukunftspläne darlegen und mich auf eine zukünftige Berufsbewerbung vorbereiten.

Wegen des Coronavirus konnten einige Dinge des P-Seminars nicht stattfinden: Besichtigungen dieser 4 Kinderkliniken wurden abgesagt: Cnopfsche Kinderklinik in Nürnberg, Haunersche Kinderklinik in München, Klinikum Dritter Orden in Nymphenburg und die private Heckscher Kinderklinik in München. Auch wollte Frau Dr. Eva Umlauf, eine Kinderpsychologin, uns einen Einblick in die Gedanken der jungen Patienten geben, was nicht zustande gekommen ist.