\documentclass[../../main.tex]{subfiles}

Im folgenden Beitrag werde Ich meine Ideen und Optionen zum Thema „Spiele – Digital“ mit Berücksichtigung auf technische Einschränkungen und andere Faktoren aufführen.

Vorweg liste ich die theoretisch möglichen Optionen auf. Diese sind aufzugliedern in Hardware und Software. Im Bereich der Hardware ist für die Maschine, auf der die Spiele laufen, ist sowohl ein PC, als auch eine Konsole (Xbox, Switch, etc.) machbar. Außerdem stellt sich die Frage, welche Eingabemethode angeboten werden soll. Hierfür bieten sich u. A. Controller oder Tastatur mit Maus an. Bei der Aufgabe der Spiele ist so gut wie alles machbar. Anbieter hierfür wären bei Konsolen deren Hersteller (Microsoft, Nintendo, etc.) und bei PCs eine Reihe an Spielehändlern wie Steam, Epic Games, etc.

Welche Option ist also die Richtige? Um diese Frage zu beantworten, müssen einige Faktoren beachtet werden. Zur Entscheidungsfindung sind Statistiken v. a. im Bereich Spielesortiment besonders wichtig. Mithilfe der Informationen in Hinsicht auf Spieleranzahl und Bewertungen kann man sich auf eine bessere Auswahl einschränken. Zusätzlich sind Umfragen für außergewöhnlichere Themen und Fragen Bezüglich der Hardware besonders wichtig. Diese können mit Klinikspezifischen Fragen erstellt werden, und sind somit die Ausschlaggebendste Quelle.

Des Weiteren gibt es Dinge, die bei Klinikpatienten zu berücksichtigen sind. Zum einen befasse ich mich mit dem Thema der Inklusion und Barrierefreiheit. Um körperlich eingeschränkte Menschen einzubinden, müssen Eingabemethoden angepasst werden (z. B. Sprachsteuerung). Zugleich muss das Problem der Einsamkeit des Patienten angegangen werden. Ein möglicher Lösungsansatz wären Online-Spiele, die mit Freunden gespielt werden können. Überdies wird das Spieleangebot von kurz untergebrachten Patienten selten angenommen, da sie sich in einer stressigen Situation befinden. Sogenannte „Hyper Casual Games“ sind ein Mittel, um diesen Personen Stress abzunehmen.

Kommen wir nun zu meinen persönlichen themenbezogenen Ideen. In Bezug auf Interoperabilität mit anderen Aufgaben der Gestaltung der Klinik gibt es die Möglichkeit, die Hardware zu teilen. Das Thema „Spiele – Digital“ für die Altersgruppe 6-11 wird mit Sicherheit viele Ähnlichkeiten mit meinem haben. Das gleiche gilt für das Thema „Unterhaltung“ – hier bieten sich sowohl PC als auch Xbox als Mediengeräte an.

Im Hinblick auf die Hardware bietet sich an, die Ausstattung zu erweitern. Pro zwei-Betten-Zimmer befindet sich ein Fernseher, den sich beide Patienten teilen müssen. Einen weiteren zu installieren ist ein Weg, die Erfahrung zu verbessern. Des Weiteren könnte man ein Konsolen-Ausleihsystem mit umfassender Auswahl einführen, um den Anforderungen jedes Besuchers gerecht zu werden. 

Zum Thema Software gibt es die Gelegenheit, eine spezielle Benutzeroberfläche zu erstellen, welche mit dem Angebot der Unterhaltung vereinbar ist. Somit können Nutzer problemlos im System navigieren.

Schlussendlich kommen wir zum Fazit. Es gibt eine Reihe an realistischen Optionen, Besuchern Digitale Spiele als Freizeitaktivität zu bieten. Man kann ein vielseitiges System anfertigen, das in Einklang mit anderen Aufgaben der Gestaltung der Klinik steht. Trotzdem muss man bei der finalen Entscheidung immer auf Umfragen und Einschränkungen achten.
